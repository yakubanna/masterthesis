\chapter{Основные понятия и обзор литературы}\label{chap1}

В настоящей главе излагаются общие принципы методов машинного обучения и общая характеристика задач оптимального управления. 


%%%%%%%%%%%%%%%%%%%%%%%%%%%%%%%%%%%%%%%%%%%%%%%%%%%%%%%%%%%%%%%%%%%%%%%%%%%%%%%%
\section{Методы машинного обучения}\label{1sec:machine-learning}
%%%%%%%%%%%%%%%%%%%%%%%%%%%%%%%%%%%%%%%%%%%%%%%%%%%%%%%%%%%%%%%%%%%%%%%%%%%%%%%%

Машинное обучение - это раздел искусственного интеллекта (ИИ), которая изучает методы создания алгоритмов, которые способны обучаться. Машинное обучение дает системам возможность автоматически учиться и совершенствоваться на основе опыта без явного алгоритма работы программы. В машинном обучении алгоритмы «обучаются» находить закономерности и особенности в огромных объемах данных, чтобы принимать решения и делать прогнозы на основе новых данных. Чем лучше алгоритм, тем точнее будут решения и прогнозы. 

Сегодня примеры машинного обучения можно встретить повсеместно. Например голосовые помощники воспроизводят музыку  по команде и ищут в интернете ответы на наши запросы. Веб-сайты рекомендуют продукты, фильмы и песни на основе того, что мы покупали, смотрели или слушали раньше. Пока мы это делаем, роботы пылесосят наши полы. Детекторы спама предотвращают попадание нежелательных писем в наши почтовые ящики. Системы анализа медицинских изображений помогают врачам определять опухоли, которые они могли пропустить. И даже первые беспилотные автомобили уже используются в качестве такси.

По мере того, как большие данные становятся все больше, вычисления становятся все более мощными и доступными, а специалисты по обработке данных продолжают разрабатывать более эффективные алгоритмы, машинное обучение будет улучшать нашу жизнь. 

Существует несколько основных видов задач машинного обучения.
\begin{enumerate}
	\item \textbf{Обучение с учителем} 
	
	Наиболее распространенный вид задач. Каждый элемент выборки представляет собой пару «объект, ответ». Требуется найти зависимость ответов от описаний объектов и построить алгоритм, принимающий на входе описание объекта и выдающий на выходе ответ. Функционал качества обычно определяется как средняя ошибка ответов, выданных алгоритмом, по всем объектам выборки. В рамках данного вида задач выделяются следующие подзадачи:
	\begin{itemize}
		\item  {\it Задача классификации.} Состоит в получении категориального ответа на основе набора признаков. Имеет конечное количество ответов (часто в виде «да» или «нет»). Например является ли животное на фотографии котом.
		\item  {\it Задача регрессии.} Состоит в прогнозировании вещественного числа на основе набора признаков. Например цену на квартиру на основе ее характеристик. 
		\item  {\it Задача ранжирования.} Отличается тем, что ответы надо получить сразу на множестве объектов, после чего отсортировать их по значениям ответов. Часто возникает в поисковых системах. 
		\item  {\it Задача прогнозирования.}  В рамках данной задачи объектами являются данные за временной интервал. Алгоритм же должен предсказать данные в следующие моменты времени. Часто встречается в рамках анализа будущей стоимости ценных бумаг.
	\end{itemize}
	
	\item \textbf{Обучение без учителя.}
	
	Обучение без учителя включает в себя класс задач обработки данных, в которых известны только описания множества объектов (обучающей выборки). В рамках данной задачи требуется найти зависимости: закономерности, внутренние взаимосвязи, зависимости, которые существуют между объектами. В рамках данного класса существует несколько основных подклассов:
	\begin{itemize}
		\item  {\it Задача кластеризации.} Основная цель заключается в  распределение данных на группы (кластеры). Например разделение людей по уровню платежеспособности.
		\item  {\it Задача уменьшения размерности.} Состоит в сведении большого числа признаков к меньшему, для удобства их последующего использования и визуализации.
	\end{itemize}
	
	\item \textbf{Частичное обучение.}
	
	Частичное обучение предлагает золотую середину между обучением с учителем и обучением без учителя. Каждый объект выборки представляет собой пару «объект, ответ», но ответы известны только для части объектов. 
	
	\item \textbf{Обучение с подкреплением (reinforcement learning).}
	В рамках данного класса объектами являются пары «ситуация, принятое решение», ответами же являются значения функционала качества, характеризующего правильность принятых решений. Часто используется в обучении роботов.
\end{enumerate}


Есть четыре основных шага для решения задачи методом машинного обучения:
\begin{enumerate}
	\item \textbf{Выбрать и подготовить набор данных для обучения.} 
	
	Обучающие данные (выборка) -- это набор данных, которые модель машинного обучения получает для решения поставленной задачи. Обычно выборка делится на тренировочную и тестовую. Тренировочные используются для обучения алгоритма, а тестовое для проверки его качества. Он извлекает из них признаки, на основе которых выбираются оптимальные параметры модели, при которых функционал качества принимает оптимальное значение.
	
	Для построения хорошего алгоритма данные для обучения должны быть правильно подготовлены -- перемешены случайным образом, из них должны быть удалены дубликаты,  устранен дисбаланс и смещение, которые могут влиять на обучение. 
	
	\item \textbf{Выбрать алгоритм.}
	На втором шаге в зависимости от класса, к которому относится исходная задача, из перечня алгоритмов машинного обучения выбирается один или множество, которые будут использоваться для ее решения. Например для обучения с учителем может быть выбран случайный лес, логистическая регрессия и другие.
	
	\item \textbf{Обучение алгоритма.}
	Обучение алгоритма - это итеративный процесс. Он включает в себя прогон переменных через алгоритм, сравнение выходных данных и правильных ответов. На основе полученных данных корректируются веса таким образом, чтобы алгоритм давал более точный результат. Далее этот шаг многократно повторяется, пока не будет достигнута необходимая точность. Полученный в результате точный алгоритм представляет собой {\it модель} машинного обучения.
	
	\item \textbf{Использование и улучшение модели.}
	Последним шагом является использование модели на новых данных и, в лучшем случае, повышение ее точности и эффективности с течением времени. Откуда будут поступать новые данные зависит от решаемой проблемы. Например, модель машинного обучения, предназначенная для выявления спама, будет принимать сообщения электронной почты, тогда как модель машинного обучения, которая управляет роботом-пылесосом, будет принимать данные, полученные в результате реального взаимодействия с перемещенной мебелью или новыми объектами в комнате. 
	
	
\end{enumerate}

 
%%%%%%%%%%%%%%%%%%%%%%%%%%%%%%%%%%%%%%%%%%%%%%%%%%%%%%%%%%%%%%%%%%%%%%%%%%%%%%%%
\section{Общая постановка и классификация задач оптимального управления}\label{1sec:optimal-control-tasks}
%%%%%%%%%%%%%%%%%%%%%%%%%%%%%%%%%%%%%%%%%%%%%%%%%%%%%%%%%%%%%%%%%%%%%%%%%%%%%%%%


Целью математической оптимизации является поиск лучшего или {\it оптимального} решения среди всевозможного набора решений, причем оптимальность его определяется через заданную функцию. Решения делятся на допустимые и недопустимые в зависимости от того, соответствуют ли они дополнительным условиям. Данные условия формально описываются в виде функций ограничений, которые накладываются на поставленную задачу. Область математической оптимизации включает в себя множество различных классов задач, которые будут кратко описаны ниже.

Исторически оптимизация отождествлялась с программированием, так как раньше слова "программа" было синонимом детерминированного плана. По этой причине многие классы задач оптимизации получили названия, в которых содержится слово «программа» или «программирование». Например термин "линейная программа" (LP), который является синонимом задачи линейной оптимизации. Даже крупнейшее сообщество математической оптимизации на протяжении десятилетий носило название «Сообщество математического программирования». Однако в 2011 году оно сменило свое название на «Общество математической оптимизации» (MOS), хоть их главный журнал «Математическое программирование» сохранил свое имя. 

В общей постановке задачи оптимального управления существует 5 основных характеристик:
\begin{enumerate}
    \item \textbf{Время.} 
    
    Существует два типа задач оптимального управления: тех, которые рассматриваются на \textit{непрерывном} промежутке времени $T=|t_0, t_f|$ и те, у которых время задается \textit{дискретным} образом: $T = {t_1, t_2, ...,t_N}$. Первое часто используется в биологических задачах, второе же в теории игр. Кроме того задача может ставиться на бесконечном временном интервале или с фиксированным временем окончания процесса. Во втором случае момент окончания называется \textit{горизонтом планирования}.
    
    \item \textbf{Состояние и математическая модель системы.} 
    
    Аналогично времени состояния системы $X$ может принадлежать конечномерному пространству $R^n$ или бесконечномерному. В первом случае задача оптимизации называется {\it конечномерной} , во втором {\it бесконечномерной} или же  задачей {\it в функциональных пространствах}. Кроме того пространство может быть непрерывным или дискретным, что тоже соответствует классификации задач оптимального управления на {\it дискретные} и {\it непрерывные}. 
    
    Динамика же изучаемого процесса моделируется чаще всего дифференциальными уравнениями: 
    
    \begin{equation} \label{eq_oc:diff}
    \dot{x}(t)=f(x(t),u(t),t),
    \end{equation}
    или разностными уравнениями: 
    \begin{equation} \label{eq_oc:razn}
    {x(k+1)}=f(x(k),u(k),k), k=0,1,...,
    \end{equation}
    где $n$-вектор $x$ -- это состояние системы, $r$-вектор $u$ -- управление, функция задана $f: R^n \times R^r \times R \rightarrow R^n$. Число $n$ называется порядком системы управления, r -- числом входов. 
    
    В зависимости от функции $f$ есть два основных класса задач:
    \begin{itemize}
    	\item  {\it Линейного программирования.} Это класс задач, где функция $f$ и функция ограничений $g$ являются линейными. 
    	\item {\it Квадратичного программирования.} Это класс задач, где функция ограничений $g$ является аффинной. А функция $f$ является квадратичной и имеет общий вид $f = c^T + \frac{1}{2} x^T H x$, где $H$ -- симметричная матрица.
    \end{itemize}
    
    \item \textbf{Класс управлений и ограничения на них.}
    
    В задачах оптимального управления четко указывается класс функций  непрерывного процесса управления , из которого выбираются управления. Это могут быть: кусочно-непрерывные, гладкие, измеримые,  импульсные функции и т.д. Также задается множество $U \in R^r$ -- множество допустимых значений управления. Как правило $U$ — компакт.
    
    Кусочно-непрерывная (измеримая, дискретная и тд.) функция $u(·) = (u(t), t \in [t_0 , t_N ])$ называется \textbf{доступным управлением}, если 

\begin{equation} \label{eq_oc:diff}
	u(t) \in U, t \in [t_0 , t_N].
\end{equation}

Аналогичное определение имеет место для дискретных систем управления.


    \item \textbf{Ограничения на фазовую траекторию.}


Аналогично размерности переменной решения размерность функций ограничений может быть конечной или бесконечной. Если присутствует бесконечное число ограничений неравенства, в то время как переменная решения конечномерна, то задача оптимизации называется полубесконечной. Этот класс часто бывает в робастной оптимизации, где нужно найти лучший выбор переменной решения, которая удовлетворяет ограничениям для всех возможных значений неизвестного, но ограниченного возмущения.

 Ограничения на переменные состояния могут накладываться в начальный момент времени $t_0$ :

\begin{equation} \label{eq_oc:diff}
	x(t_0) \in X_0,
\end{equation}
и в конечный момент времени $t_N$:

\begin{equation} \label{eq_oc:diff}
	x(t_N) \in X_N
\end{equation}
такие ограничения называются \textit{терминальными}. Так же существуют ограничения в изолированные моменты из промежутка управления $t_i \in [t_0 , t_N], i = 1, N$:
$$x(t_i) \in X_i , i = 1, N$$
такие ограничения называются \textit{промежуточными фазовыми ограничениями}.

Кроме того могут быть заданы ограничения на всем промежутке управления — фазовые ограничения: $$x(t) \in X(t), t \in [t_0 , t_N ],$$
где $X_0 , X_N , X_i , i = 1, m, X(t), t \in [t_0 , t_N ]$, — заданные множества пространства состояний.

Доступное управление $u(·) = (u(t), t \in [t_0 , t_N ])$ называется \textbf{допустимым (или, программой)}, если оно порождает траекторию $x(·)$, удовлетворяющую всем ограничениям задачи.
    
    
    \item \textbf{Критерий качества задач оптимального управления.}

 Множество допустимых управлений  содержит более одного элемента, поэтому возникает необходимость сравнивать их между собой. Для этого вводится функционал $J(u)$, называемый критерием качества, и выбирается операция минимизации или максимизации этого функционала, результат которой определяет наилучшее (оптимальное) управление. В теории оптимального управления различают четыре типа критериев качества:
\begin{enumerate}
    \item критерий качества Майера (терминальный критерий)
    \begin{equation} \label{eq_op:Mayers} J(u) = \phi(x(t_N)),
    \end{equation}
    \item критерий качества Лагранжа (интегральный критерий)
    \begin{equation} \label{eq_op:Lagranzh} \int_{t_0}^{t_N}f_0(x(t),u(t),t)dt,
    \end{equation}
    \item критерий качества Больца
    
    \begin{equation} \label{eq_op:Bolc} J(u) = \phi(x(t_N)) + \int_{t_0}^{t_N}f_0(x(t),u(t),t)dt,
    \end{equation}
    \item задачи быстродействия
    \begin{equation} \label{eq_op:Bystr}J(u)=t_N - t_0 \rightarrow \min.
    \end{equation}
\end{enumerate}
Все критерии качества эквивалентны между собой.

Допустимое управление $u_0 (·)$ называется \textbf{оптимальным управлением (оптимальной программой)}, если на нем критерий качества достигает экстремального значения (min или max): $$J(u_0) = extr J(u),$$
где минимум (максимум) берется по всем допустимым управлениям.

\end{enumerate}



%%%%%%%%%%%%%%%%%%%%%%%%%%%%%%%%%%%%%%%%%%%%%%%%%%%%%%%%%%%%%%%%%%%%%%%%%%%%%%%%
\section{Постановка задачи}\label{1sec:task}
%%%%%%%%%%%%%%%%%%%%%%%%%%%%%%%%%%%%%%%%%%%%%%%%%%%%%%%%%%%%%%%%%%%%%%%%%%%%%%%%

Основной объект исследования настоящей работы — дискретная стационарная нелинейная динамическая система, вида:
\begin{equation}
	x_{k+1} = f(x_k, u_k), \forall k = \overline{0, N-1}, 
\end{equation}

где $x_k$ -- состояние системы, $\u_k$ -- управление в дискретные моменты времени $k$.

Если задано начальное состояние $\overline{x_0}\in \mathbb{R^n}$ и последоватлеьность управляющих воздействий $u_0, \dots , u_{N-1} \in \mathbb{R^r}$, то можно вычислить будущее состояние системы в любой момент времени $k=\overline{1, N}$ , то есть найти траекторию 

\begin{equation}
	x(\overline{x_0}, u) = {x_k(\overline{x_0}, u), k = \overline{0, N-1}},  
\end{equation}

соответсвующих начальному состоянию $x_0$ и управлению 

\begin{equation}
	u = {u_0, \dots, u_{N-1}}.
\end{equation}

В рамках данной работы будем рассматривать задачу маятника. В ней текущее состояние состоит из параметра:

\begin{itemize}
	\item угол отклонения маятника от вертикальной оси ($\phi$)
\end{itemize}

На него накладываются следующие ограничения:
\begin{equation}
	\begin{aligned}
		\phi_{min} \leq \phi_k \leq \phi_{max} \\
		k = \overline{0, N-1}.
	\end{aligned}
	\label{eq:pend}
\end{equation}

Всего существует одно  управление $u = \pm 1$, а переход из одного состояния происходит в соответствии с законами физики:
\begin{equation}
	\begin{aligned}
		\ddot{\phi} + \omega^2 \phi = u 
	\end{aligned}
\end{equation}

где $\omega^2 = \dfrac{g}{l}$ -- это постоянный коэффициент. 
Пример данный задачи можно посмотреть на рисунке ~\ref{fig:pend}. 
\begin{figure}[h]
	\centering
	\includegraphics[scale=0.3]{mayat.png}
	\caption {Модель маятника}
	\label{fig:pend}
\end{figure}


%%%%%%%%%%%%%%%%%%%%%%%%%%%%%%%%%%%%%%%%%%%%%%%%%%%%%%%%%%%%%%%%%%%%%%%%%%%%%%%%
\section{Управление с прогнозирующими моделями (MPC)}\label{1sec:mpc}
%%%%%%%%%%%%%%%%%%%%%%%%%%%%%%%%%%%%%%%%%%%%%%%%%%%%%%%%%%%%%%%%%%%%%%%%%%%%%%%%


Подход управления с прогнозирующими моделями начал развиваться в начале 60-х годов XX века для управления процессами и оборудованием в нефтехимическом и энергетическом производстве, для которых применение традиционных методов синтеза было крайне затруднено в связи с исключительной сложностью их математических моделей. В последнее время область применения MPC значительно расширилась, охватывая технологические отрасли и экономику при управлении производством, при решении задач управления запасами и портфелем ценных бумаг.

Основным достоинством MPC-подхода, определяющим его успешное использование в практике построения и эксплуатации систем управления, служит относительная простота базовой схемы формирования обратной связи, сочетающаяся с высокими адаптивными свойствами. Последнее обстоятельство позволяет управлять многомерными и многосвязными объектами со сложной структурой, оптимизировать процессы в режиме реального времени в рамках ограничений на управляющие и управляемые переменные, учитывать неопределенности в задании объектов и возмущений. Кроме того, возможен учет запаздываний, поскольку зачастую решение об управлении принимается в момент времени t – h, а реализация этого решения происходит в момент времени t.

Управление по прогнозирующей модели, в англоязычной литературе носящее название Model Predictive Control (MPC), является одним из современных методов теории управления \cite{mpcIn, mpcIn2}. Популярность МРС на практике обусловлена тем, что метод применим к многосвязным объектам управления, в том числе нелинейным, учитывает ограничения на управляющие и фазовые переменные, принимает во внимание качественные требования к процессу
управления. Классической областью применения MPC до недавнего времени были задачи стабилизации и слежения в технических приложениях. Теоретические основы метода для задач стабилизации получили строгое обоснование в работах \cite{mpcIn, mpcIn2}.


\definition{Управление по прогнозирующей модели (англ. MPC) -- современный подход для решения задач стабилизации.}

MPC основывается на последовательном, в каждый момент времени, решении прогнозирующих задач оптимального управления (predictive optimal control problems) с конечным горизонтом управления, сформулированных для математической модели управляемого объекта, начальное условие которой совпадает с измеренным состоянием объекта. Значение оптимального программного управления прогнозирующей задачи на левом конце промежутка управления используется для управления объектом в текущий момент времени и до тех пор, пока не будет получено и обработано следующее измерение состояния.

Управление, которое подается на объект в описанном процессе, представляет собой \textit{обратную связь} (оно зависит от измеряемых состояний), свойства которой зависят от конкретного вида прогнозирующей задачи оптимального управления и целей управления объектом (например, стабилизация, регулирование, слежение и др.).

Методы МРС предполагают построение стабилизирующей обратной связи на основе повторяющегося в каждый текущий момент времени решения задач оптимального управления (ОУ)\cite{mpc}. Для того, чтобы учесть практическую невозможность мгновенного вычисления решения задач ОУ, предполагается, что состояния среды обрабатываются (измеряются) в дискретные моменты времени $\tau \in {0, h, 2h, \dots}$, где $h > 0$ – период квантования,
превосходящий время решения задач ОУ. Соответственно, будет строиться дискретная обратная связь, что позволяет определить решение замкнутой системы при заданном начальном состоянии как последовательное решение уравнения
$\ddot{x} = f(x(t), u(\tau)), x(\tau) = x(\tau - 0) $ на интервалах $t \in [\tau, \tau + h[$.

Общая идея МРС состоит в решении в каждый момент $\tau$ так называемой прогнозирующей задачи ОУ с конечным горизонтом $T = Nh$ ( $N$ – натуральное число), в которой начальное условие для прогнозирующей модели совпадает с измеренным состоянием $x(\tau)$ объекта управления. В качестве прогнозирующей модели выбирается математическая модель объекта управления, которая может отличаться: это может быть линеаризация,
или детерминированная модель, в которой не учитываются возмущения, немоделируемая динамика, другие неопределенности.  В данной работе считается, что прогнозирующая модель совпадает с исходной моделью. Такой подход называется номинальным МРС. 

Базовый алгоритм МРС состоит в следующем: для каждого $t$ выполнить
\begin{enumerate}
	\item измерить текущее состояние $x(t)$ объекта управления;
	\item решить задачу, получить оптимальное программное управление $u$
	\item подать в среду управляющее воздействие.
\end{enumerate}

Построенная в результате применения данного алгоритма функция 
$u_{MPC} (t),$  $ t \ge 0$ , является реализацией дискретной обратной связи вида $u = u^0(\tau)$ , вдоль траектории измерений, реализовавшейся в конкретном процессе управления.  Она обеспечивает асимптотическую устойчивость решения замкнутой системы при ряде дополнительных условий. В частности, известно, что при недостаточно больших горизонтах управления $T$ система может оказаться неустойчивой, поэтому авторами \cite{mpcIn2} получены нижние оценки параметра $T$, гарантирующие асимптотическую устойчивость. Нужно отметить,
что при таком подходе горизонт управления может оказаться достаточно большим, что отрицательно скажется на трудоемкости решения задачи. Исторически первый подход состоит в том, чтобы дополнить задачу  терминальным ограничением. Недостатками данного подхода являются: 1) при коротких горизонтах планирования прогнозирующая задача ОУ может не иметь решения, 2) двухточечные задачи ОУ являются самыми сложными с вычислительной точки зрения. Устраняют перечисленные недостатки подходы \cite{mpcIn2}, в которых задача дополняется
терминальным ограничением , выбирается критерий качества типа Больца (см. главу 1). Асимптотическая устойчивость замкнутой системы в подходах \cite{mpcIn2} гарантируется в том случае, если существует некоторое множество $S$, на котором найдется такая локальная обратная связь $U$, что: 1) множество $S$ является положительно инвариантным для системы $\ddot{x} =  f (x, k (x)), x \in S$ ; 2) при всех $x \in S$ имеет место включение $k(x) \in U$ .


В последние годы исследования в рамках МРС сместились в сторону решения задач, в которых целью управления является максимизация или минимизация экономического критерия (издержки и другое). Теперь   основные свойства МРС, сделавшие его популярным на прктике можно сформулировать как:
\begin{enumerate}
	\item критерий качества в задаче оптимального управления  позволяет учитывать экономические требования к процессу управления;
	\item учитываются жесткие ограничения на фазовые и управляющие переменные;
	\item метод применим к нелинейным и многосвязным системам.
\end{enumerate}



%%%%%%%%%%%%%%%%%%%%%%%%%%%%%%%%%%%%%%%%%%%%%%%%%%%%%%%%%%%%%%%%%%%%%%%%%%%%%%%%
\section{Выводы}\label{1sec:conc}
%%%%%%%%%%%%%%%%%%%%%%%%%%%%%%%%%%%%%%%%%%%%%%%%%%%%%%%%%%%%%%%%%%%%%%%%%%%%%%%%

В рамках данной главы была представлена общая характеристика и рассмотрена классификация задач оптимального управления. На основе чего была выбрана и поставлена задача, решаемая в рамках данной работы. Также были рассмотрены основные принципы и классификация методов машинного обучения. В результате из всех разновидностей выбран метод обучения с подкреплением для решения поставленной задачи. Более подробно он рассматривается в следующей главе.