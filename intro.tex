\chapter*{Введение}
\addcontentsline{toc}{chapter}{ВВЕДЕНИЕ}


Оптимизация и управление на основе данных (data-driven optimization and control) – новое направление исследований в теории оптимизации и теории управления  \cite{dyn1, dyn2}. Актуальность и интерес к этому направлению вызваны двумя факторами.

Во-первых, это рост объемов данных, развитие и доступность различных измерительных устройств и сенсоров, позволяющих записывать ранее недоступные данные, и, соответственно, бурное развитие алгоритмов и систем их обработки. 

Во-вторых, объекты управления или оптимизации в прикладных исследованиях становятся все сложнее, и, соответственно, растут размерности и сложность их математических моделей, необходимых для применения существующих методов управления и оптимизации. Также возрастает число ограничений, накладываемых на переменные моделей, и зачастую эти ограничения связаны с безопасностью функционирования системы, т. е. не могут быть нарушены или ослаблены. 

В настоящей работе будут исследоваться задачи управления динамическими объектами, в частности одна из центральных проблем теории управления -- задача стабилизации. Одним из набирающих популярность в промышленных приложениях подходов к решению этой задачи является так называемый метод управления по прогнозирующей модели (Model Predictive Control – MPC) \cite{mpcIn, mpcIn2}. Он основан на решении в реальном времени специально подобранной (прогнозирующей) задачи оптимального управления, которая в своей формулировке содержит математическую модель объекта управления в пространстве состояний, различные ограничения на состояния и управления и начальное условие, совпадающее с текущим состоянием объекта стабилизации.  Для успеха реализации стратегии МРС, с одной стороны, требуется модель, описывающая процесс управления с высокой точностью, с другой же стороны, для решения задач оптимального управления в реальном времени модель (она задает ограничения-равенства в задаче) должна быть достаточно простой, иначе существующие численные методы решения оптимизационных задач могут не построить решение за требуемое время.

Другим примером может служить классическая задача из теории оптимального управления --- задача синтеза оптимальной системы \cite{npont} (построение оптимальной обратной связи в задаче оптимального управления). Эта задача до сих пор не решена в классической постановке (построение обратной связи как функции всех позиций системы управления), однако для нее существует подход, называемый оптимальным управлением в реальном времени \cite{ngab} и близкий по технике методу управления в реальном времени. В связи с приведенными выше факторами, а также описанными примерами, естественной и привлекательной становится следующая идея: 1) непосредственно использовать технологические или экспериментальные данные, полученные в результате наблюдений за поведением динамических систем, в формулировке задачи управления, исключая предварительный шаг идентификации системы по этим данным (этап математического моделирования); 2) привлекать алгоритмы обработки больших данных с целью повышения эффективности схем управления. Один из подходов, реализующих идею 1), можно найти в работе \cite{kost}. Еще одним метод, в котором реализована идея 1), является обучение с подкреплением. 

В рамках данной работы проводится общий обзор управления по прогнозирующей модели. Также сделан обзор основных свойств методов машинного обучения и выделен подкласс методов, подходящих для решения задач оптимального управления, -- обучение с подкреплением. В первой главе поставлена классическая задача оптимального управления, требующая решение.  Во второй и третей главе  соответственно описаны алгоритмы обучения с подкреплением и управления по прогнозирующей модели для решения поставленной задачи. В четвертой главе приведены практические аспекты их реализации и полученные результаты, произведен сравнительный анализ двух методов.

