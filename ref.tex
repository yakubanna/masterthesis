\chapter*{Реферат}\label{ref}
\addcontentsline{toc}{chapter}{РЕФЕРАТ}

Магистерская диссертация, \pageref{lastpage}~стр.,  3 рис., 8 таблиц, 13 источников.

\textbf{Ключевые слова:} МАШИННОЕ ОБУЧЕНИЕ, ОБУЧЕНИЕ С ПОДКРЕПЛЕНИЕМ, ЗАДАЧИ ОПТИМАЛЬНОГО УПРАВЛЕНИЯ, НЕЙРОННЫЕ СЕТИ, УПРАВЛЕНИЕ С ПРОГНОЗИРУЮЩИМИ МОДЕЛЯМИ.

\textbf{Объект исследования} -- обучение с подкреплением и управление с прогнозирующими моделями в рамках задач оптимального управления. 

\textbf{Цель работы} -- решить задачу оптимального управления методом обучения с подкреплением и управлением с прогнозирующими моделями, на основе чего провести сравнительный анализ.

\textbf{Результаты исследования:} 
\begin{enumerate}
	\item Изучены основные задачи оптимального управления, выбрана и поставлена задача.
	\item Проведен обзор и выбран метод обучения с подкреплением для решения задачи.
	\item Решена поставленная задача с помощью выбранного метода обучения с подкреплением.
	\item Проведен обзор метода решения задач с помощью управления с прогнозирующими моделями.
	\item Решена задача с помощью метода с прогнозирующими моделями.
	\item Проведен сравнительный анализ решений, полученных вышеперечисленными методов. 
\end{enumerate}

\textbf{Область применения} -- выбор решений для более сложных задач оптимального управления.

\newpage

	\section*{\centerline{РЭФЕРАТ}} 

Магiстарская дысертацыя, \pageref{lastpage}~ст.,  3 мал., 8 таблiц, 13 крынiц.

\textbf{Ключавыя словы:} МАШЫННАЕ НАВУЧАННЕ, НАВУЧАННЕ З ПАДМАЦАВАННЕМ, ЗАДАЧЫ АПТЫМАЛЬНАГА КІРАВАННЯ, НЕЙРОННАЯ СЕТКА, КІРАВАННЕ З ПРАГНАЗУЕМЫМІ МАДЭЛЯМІ

\textbf{Аб’ект даследавання} -- навучанне з падмацаваньнем і кіраванне з прагназуюць мадэлямі ў рамках задач аптымальнага кіравання.

\textbf{Мэта працы} -- рашыць задачу аптымальнага кіравання метадам навучання з падмацаваньнем і кіраваннем з прагназуючымі мадэлямі, на аснове чаго правесці параўнальны аналіз.

\textbf{Вынiкi даследавання:} 
\begin{enumerate}
	\item Даследаваны асноўныя задачы аптымальнага кіравання, абрана і пастаўлена задача.
	\item Праведзены агляд і абран метад навучання з падмацаваньнем для вырашэння задачы.
	\item Рашана пастаўленая задача з дапамогай абранага метаду навучання з падмацаваньнем.
	\item Праведзены агляд метаду рашэння задач з дапамогай кіравання з прагназуючымі мадэлямі.
	\item Рашана задача з дапамогай метаду з прагназуючымі мадэлямі.
	\item Праведзены параўнальны аналіз рашэнняў, атрыманых вышэйпералічанымі метадамі.
\end{enumerate}

\textbf{Вобласць прымянення} -- выбар рашэнняў для больш складаных задач аптымальнага кіравання.

\newpage



\section*{\centerline{ABSTRACT}} 

Master thesis \pageref{lastpage}~p., 7 pictures, 1 tables, 15 sources

\textbf{Key words:} MACHINE LEARNING, REINFORCEMENT LEARNING, OPTIMAL CONTROL PROBLEMS, NEURAL NETWORKS, MODEL PEDICTIVE CONTROL.

\textbf{Object of research} -- reinforcement learning and control with predictive models in the framework of optimal control problems.

\textbf{The aim of the work} is to solve the problem of optimal control by reinforcement learning and model predictive control. Then make compare analysis based on it.

\textbf{The results:} 
\begin{enumerate}
	\item The main tasks of optimal control have been studied, the task has been selected and set.
	\item A review was carried out and a reinforcement learning method was selected for solving the problem.
	\item The task was solved using the selected reinforcement learning method.
	\item A review of the method for solving problems using model predictive control method is carried out.
	\item The problem is solved using model predictive control.
	\item A comparative analysis of the solutions obtained by the above methods is carried out.
\end{enumerate}

\textbf{Field of application} -- the choice of solutions for complex problems of optimal control.